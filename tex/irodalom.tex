%----------------------------------------------------------------------------
%Irodalomjegyzék
%----------------------------------------------------------------------------
\begin{thebibliography}{9}
%Könyvek,cikkek,tanulmányok
\bibitem{Chollet}
	François Chollet,
	\textbf{Deep Learning with Python},
	Manning Publications,
	2018.

\bibitem{neural2006}
	Altrichter Márta \& Horváth Gábor \& Pataki Béla \& Strausz György \& Takács Gábor \& Valyon József,
	\textbf{Neurális hálózatok},
	Panem,
	2006,
	[Szakkönyv],

\bibitem{mccullogh1943}
McCulloch, W.S. \& Pitts, W.,
\textbf{A Logical Calculus of Ideas Immanent in Nervous Activity},
Bulletin of Mathematical Biophysics,
1943,
doi:10.1007/BF02478259.

%adatforrások

%jogszabályok

%Kézikönyvek

\bibitem{web:PlaidML}
\textbf{PlaidML -- Home},
2019. október 20.,
Vertex.AI.,
\newline\url{https://vertexai-plaidml.readthedocs-hosted.com/en/stable/}

\bibitem{web:ngraph_intro}
\textbf{Introduction --- Documentation for the {nGraph} Library and Compiler stack},
2019,
\newline\url{https://ngraph.nervanasys.com/docs/latest/introduction.html},
[meglátogatva: 2019. október 08.]

\bibitem{web:OpenVINO}
\textbf{OpenVINO\textsuperscript{\texttrademark}\space toolkit Documentation},
2019,
\newline\url{https://docs.openvinotoolkit.org/2019_R3/index.html},
[meglátogatva: 2019. november 4.]

%Internetes adatgyűjtés

	\bibitem{Nielsen2015}
	Michael A. Nielsen,
	\textbf{Neural Networks and Deep Learning},
	Determination Press
	2015.
	\newline\url{http://neuralnetworksanddeeplearning.com/}
	[online, meglátogatva: 2019. október 30.]
	
	\bibitem{web:GoogleEdge}
	\textbf{Edge TPU: Hands-On with Google’s Coral USB Accelerator},
	\newline\url{https://heartbeat.fritz.ai/edge-tpu-google-coral-usb-accelerator-cf0d79c7ec56},
	[meglátogatva: 2019. október 23.]
	
	\bibitem{web:Keras}
	\textbf{Keras Dokumentáció},
	\newline\url{https://keras.io/},
	[meglátogatva: 2019. október 23.]
	
	%Github
\bibitem{github:nGraph}
	\textbf{NervanaSystems/ngraph: nGraph - open source C++ library, compiler and runtime for Deep Learning},
	GitHub repository,
	2019,
	\newline\url{https://github.com/NervanaSystems/ngraph},

\bibitem{github:PlaidML}
	\textbf{plaidml/plaidml: PlaidML is a framework for making deep learning work everywhere},
	GitHub repository,
	\newline\url{https://github.com/plaidml/plaidml},
	[meglátogatva: 2019. október 20.]
	
	%Wikipedia
	
	\bibitem{ wiki:constfold}
	\textbf{Constant folding --- {Wikipedia}{,} The Free Encyclopedia},
	2019,
	\newline\url{https://en.wikipedia.org/w/index.php?title=Constant_folding&oldid=914455114},
	[meglátogatva: 2019. október 09.]
	
	\bibitem{wiki:plaidml}
	\textbf{PlaidML --- {Wikipedia}{,} The Free Encyclopedia},
	2019,
	\newline\url{https://en.wikipedia.org/w/index.php?title=PlaidML&oldid=898300482},
	[meglátogatva: 2019. október 5.]
	
	\bibitem{wiki:relu}
	\textbf{Rectifier (neural networks) --- {Wikipedia}{,} The Free Encyclopedia},
	2019,
	\newline\url{ https://en.wikipedia.org/w/index.php?title=Rectifier_(neural_networks)&oldid=923288576}
	[Meglátogatva: 2019. október 29.]
	
	\bibitem{wiki:TPU}
	\textbf{Tensor processing unit --- {Wikipedia}{,} The Free Encyclopedia},
	2019,
	\newline\url{https://en.wikipedia.org/w/index.php?title=Tensor_processing_unit&oldid=923241091},
	[Meglátogatva: 2019. november 4.]
	
\end{thebibliography}