%----------------------------------------------------------------------------
%Irodalomjegyzék
%----------------------------------------------------------------------------
\begin{thebibliography}{9}
%Könyvek,cikkek,tanulmányok
\bibitem{chollet}
	François Chollet,
	\textbf{Deep Learning with Python},
	Manning Publications,
	2018.

\bibitem{neural2006}
	Altrichter Márta \& Horváth Gábor \& Pataki Béla \& Strausz György \& Takács Gábor \& Valyon József,
	\textbf{Neurális hálózatok},
	Panem,
	2006

\bibitem{mccullogh1943}
	McCulloch, W.S. \& Pitts, W.,
	\textbf{A Logical Calculus of Ideas Immanent in Nervous Activity},
	Bulletin of Mathematical Biophysics,
	1943,
	doi:10.1007/BF02478259

\bibitem{nielsen2015}
	Michael A. Nielsen,
	\textbf{Neural Networks and Deep Learning},
	Determination Press,
	2015,
	\newline{\footnotesize\url{http://neuralnetworksanddeeplearning.com/}},
	\mbox{[meglátogatva:~2019.~október~30.]}

\bibitem{agulleiro2012}
	Agulleiro, J. I., Vázquez, F., Garzón, E. M. \& Fernández, J. J., \textbf{Hybrid computing: CPU+GPU co-processing and its application to tomographic reconstruction},
	Ultramicroscopy,
	2012,
	doi:10.1016/j.ultramic.2012.02.003 

\bibitem{meredith2011}
	Meredith, J., Roth, P., Spafford, K. \& Vetter, J.,
	\textbf{Performance Implications of Nonuniform Device Topologies in Scalable Heterogeneous Architectures},
	IEEE Micro,
	2011,
	doi:10.1109/mm.2011.79 

\bibitem{li2019}
	Li, X., Wan, J., Dai, H.-N., Imran, M., Xia, M. \& Celesti, A.,
	\textbf{A Hybrid Computing Solution and Resource Scheduling Strategy for Edge Computing in Smart Manufacturing},
	IEEE Transactions on Industrial Informatics,
	2019,
	doi:10.1109/tii.2019.2899679

%adatforrások
\bibitem{spec:titan-rtx}
	\textbf{NVIDIA TITAN RTX Specs},
	Techpowerup,
	\newline{\footnotesize\url{https://www.techpowerup.com/gpu-specs/titan-rtx.c3311}},
	\mbox{[meglátogatva:~2019.~november~19.]}

\bibitem{spec:tesla-t4}
	\textbf{NVIDIA Tesla T4 Specs},
	Techpowerup,
	\newline{\footnotesize\url{https://www.techpowerup.com/gpu-specs/tesla-t4.c3316}},
	\mbox{[meglátogatva:~2019.~november~19.]}

%jogszabályok

%Kézikönyvek

\bibitem{web:cudnn}
	\textbf{cuDNN Developer Guide},
	2019,
	Nvidia,
	\newline{\footnotesize\url{https://docs.nvidia.com/deeplearning/sdk/cudnn-developer-guide/}},
	\mbox{[meglátogatva:~2019.~november~19.]}

\bibitem{web:PlaidML}
	\textbf{PlaidML -- Home},
	2019,
	Vertex.AI.,
	\newline{\footnotesize\url{vertexai-plaidml.readthedocs-hosted.com/en/stable/}},
	\mbox{[meglátogatva:~2019.~október~20.]}

\bibitem{web:ngraph_intro}
	\textbf{Introduction --- Documentation for the {nGraph} Library and Compiler stack},
	2019,
	\newline{\footnotesize\url{ngraph.nervanasys.com/docs/latest/introduction.html}},
	\mbox{[meglátogatva:~2019.~október~08.]}

\bibitem{web:OpenVINO}
	\textbf{OpenVINO\textsuperscript{\texttrademark}\space toolkit Documentation},
	2019,
	\newline{\footnotesize\url{docs.openvinotoolkit.org/2019_R3/index.html}},
	\mbox{[meglátogatva:~2019.~november~4.]}

%Internetes adatgyűjtés

\bibitem{web:GoogleEdge}
	\textbf{Edge TPU: Hands-On with Google’s Coral USB Accelerator},
	\newline{\footnotesize\url{heartbeat.fritz.ai/edge-tpu-google-coral-usb-accelerator-cf0d79c7ec56}},
	\mbox{[meglátogatva:~2019.~október~23.]}

\bibitem{web:Keras}
	\textbf{Keras Dokumentáció},
	\newline{\footnotesize\url{keras.io/}},
	\mbox{[meglátogatva:~2019.~október~23.]}

\bibitem{web:NNP}
	\textbf{Intel® Nervana™ Neural Network Processors},
	\newline{\footnotesize\url{www.intel.ai/nervana-nnp/}},
	\mbox{[meglátogatva:~2019.~november~07.]}

%\bibitem{web:Raoblog2019}
%	Naveen Rao,
%	\textbf{‘‘Accelerating with Purpose'' for AI everywhere},
%	2019,
%	\newline{\footnotesize\url{www.intel.ai/accelerating-for-ai/}},
%	\mbox{[blog, meglátogatva:~2019.~November~07.]}

%\bibitem{web:Raoblog2017}
%	Naveen Rao,
%	\textbf{Intel® Nervana™ Neural Network Processors (NNP) Redefine AI Silicon},
%	2017,
%	\newline{\footnotesize\url{www.intel.ai/intel-nervana-neural-network-processors-nnp-redefine-ai-silicon/}},
%	\mbox{[blog, meglátogatva:~2019.~November~07.]}

\bibitem{web:IntelNews2019}
	\textbf{At Hot Chips, Intel Pushes ‘AI Everywhere’},
	2019,
	\newline{\footnotesize\url{newsroom.intel.com/news/hot-chips-2019/}},
	\mbox{[meglátogatva:~2019.~November~07.]}

\bibitem{yang-nnpt}
	Andrew Yang,
	\textbf{Deep Learning Training At Scale	Spring Crest Deep Learning Accelerator (Intel® NervanaTM NNP-T)},
	2019,
	\newline{\footnotesize\url{www.slideshare.net/insideHPC/deep-learning-training-at-scale-spring-crest-deep-learning-accelerator/}},
	\mbox{[letöltve:~2019.~november~07.]}

\bibitem{yt:nnpt}
	Intel AI,
	\textbf{Introducing Intel Nervana Neural Network Processors for Training},
	2019,
	\newline{\footnotesize\url{www.youtube.com/watch?v=Wj4ifr3DDFg}},
	\mbox{[megtekintve:~2019.~november~10.]}

\bibitem{Wechsler-nnpi}
	Ofri Wechsler \& Michael Behar \& Bharat Daga,
	\textbf{Spring Hill (NNP-I 1000) Intel’s Data Center Inference Chip},
	2019,
	\newline{\footnotesize\url{www.slideshare.net/insideHPC/spring-hill-nnpi-1000-intels-data-center-inference-chip/}},
	\mbox{[letöltve:~2019.~november~07.]}

\bibitem{yt:nnpi}
	Intel AI,
	\textbf{Introducing Intel Nervana Neural Network Processors for Inference},
	2019,
	\newline{\footnotesize\url{www.youtube.com/watch?v=fTMIjQePWog}},
	\mbox{[megtekintve:~2019.~november~10.]}

	%Github

\bibitem{github:nGraph}
	\textbf{NervanaSystems/ngraph: nGraph - open source C++ library, compiler and runtime for Deep Learning},
	GitHub repository,
	2019,
	\newline{\footnotesize\url{github.com/NervanaSystems/ngraph}},

\bibitem{github:PlaidML}
	\textbf{plaidml/plaidml: PlaidML is a framework for making deep learning work everywhere},
	GitHub repository,
	\newline{\footnotesize\url{github.com/plaidml/plaidml}},
	\mbox{[meglátogatva:~2019.~október~20.]}

	%Wikipedia

\bibitem{wiki:constfold}
	\textbf{Constant folding},
	{Wikipedia}{,} The Free Encyclopedia,
	2019,
	\newline{\footnotesize\url{en.wikipedia.org/w/index.php?title=Constant_folding&oldid=914455114}},
	\mbox{[meglátogatva:~2019.~október~09.]}

\bibitem{wiki:listOfIGPU}
	\textbf{List of Intel graphics processing units},
	{Wikipedia}{,} The Free Encyclopedia,
	2019,
	\newline{\footnotesize\url{en.wikipedia.org/w/index.php?title=List_of_Intel_graphics_processing_units&oldid=926282215}},
	\mbox{[meglátogatva:~2019.~október~20.]}

\bibitem{wiki:plaidml}
	\textbf{PlaidML},
	{Wikipedia}{,} The Free Encyclopedia,
	2019,
	\newline{\footnotesize\url{en.wikipedia.org/w/index.php?title=PlaidML&oldid=898300482}},
	\mbox{[meglátogatva:~2019.~október~5.]}

\bibitem{wiki:relu}
	\textbf{Rectifier (neural networks)},
	{Wikipedia}{,} The Free Encyclopedia,
	2019,
	\newline{\footnotesize\url{ en.wikipedia.org/w/index.php?title=Rectifier_(neural_networks)&oldid=923288576}},
	\mbox{[meglátogatva:~2019.~október~29.]}

\bibitem{wiki:TPU}
	\textbf{Tensor processing unit},
	{Wikipedia}{,} The Free Encyclopedia,
	2019,
	\newline{\footnotesize\url{en.wikipedia.org/w/index.php?title=Tensor_processing_unit&oldid=923241091}},
	\mbox{[meglátogatva:~2019.~november~4.]}

\bibitem{wiki:cuda}
	\textbf{CUDA},
	{Wikipedia}{,} The Free Encyclopedia,
	2019,
	\newline{\footnotesize\url{https://en.wikipedia.org/w/index.php?title=CUDA&oldid=924918335}},
	\mbox{[meglátogatva:~2019.~november~19.]}

	%cikkek, újságok
\bibitem{tyson-hexus}
	Mark Tyson,
	\textbf{Industry sources say discrete Intel Xe cards will arrive mid-2020},
	2019.~október~16.
	\newline{\footnotesize\url{hexus.net/tech/news/graphics/135800-industry-sources-say-discrete-intel-xe-cards-will-arrive-mid-2020}}
	[megtekintve:~2019.~november~21.]

\bibitem{allan-techradar}
	Darren Allan,
	\textbf{Intel’s first Xe graphics card is officially ‘alive’ and coming for AMD and Nvidia},
	2019.~október~26.
	\newline{\footnotesize\url{www.techradar.com/news/intels-first-xe-graphics-card-is-officially-alive-and-coming-for-amd-and-nvidia}}
	[megtekintve:~2019.~november~21.]

\bibitem{cutress-anandtech}
	Dr. Ian Cutress,
	\textbf{Intel’s Xe for HPC: Ponte Vecchio with Chiplets, EMIB, and Foveros on 7nm, Coming 2021},
	2019.~november~17.
	\newline{\footnotesize\url{www.anandtech.com/show/15119/intels-xe-for-hpc-ponte-vecchio-with-chiplets-emib-and-foveros-on-7nm-coming-2021}}
	[megtekintve:~2019.~november~21.]

\bibitem{patrizio-networkworld}
	Andy Patrizio,
	\textbf{Intel targets Nvidia (again) with GPU and cross-processor API},
	2019.~november~19.
	\newline{\footnotesize\url{www.networkworld.com/article/3454497/intel-targets-nvidia-again-with-gpu-and-cross-processor-api.html}}
	[megtekintve:~2019.~november~21.]

\end{thebibliography}