%----------------------------------------------------------------------------
%Irodalomjegyzék
%----------------------------------------------------------------------------
\begin{thebibliography}{9}
%Könyvek,cikkek,tanulmányok
\bibitem{Chollet}
	François Chollet,
	\textbf{Deep Learning with Python},
	Manning Publications,
	2018.

\bibitem{neural2006}
	Altrichter Márta \& Horváth Gábor \& Pataki Béla \& Strausz György \& Takács Gábor \& Valyon József,
	\textbf{Neurális hálózatok},
	Panem,
	2006,
	[Szakkönyv],

\bibitem{mccullogh1943}
	McCulloch, W.S. \& Pitts, W.,
	\textbf{A Logical Calculus of Ideas Immanent in Nervous Activity},
	Bulletin of Mathematical Biophysics,
	1943,
	doi:10.1007/BF02478259.

\bibitem{Nielsen2015}
	Michael A. Nielsen,
	\textbf{Neural Networks and Deep Learning},
	Determination Press
	2015.
	\newline{\footnotesize\url{http://neuralnetworksanddeeplearning.com/}}
	[online, meglátogatva: 2019.~október~30.]
	
%adatforrások

%jogszabályok

%Kézikönyvek

\bibitem{web:PlaidML}
	\textbf{PlaidML -- Home},
	2019.~október~20.,
	Vertex.AI.,
	\newline{\footnotesize\url{vertexai-plaidml.readthedocs-hosted.com/en/stable/}}

\bibitem{web:ngraph_intro}
	\textbf{Introduction --- Documentation for the {nGraph} Library and Compiler stack},
	2019,
	\newline{\footnotesize\url{ngraph.nervanasys.com/docs/latest/introduction.html}}
	[meglátogatva: 2019.~október~08.]

\bibitem{web:OpenVINO}
	\textbf{OpenVINO\textsuperscript{\texttrademark}\space toolkit Documentation},
	2019,
	\newline{\footnotesize\url{docs.openvinotoolkit.org/2019_R3/index.html}}
	[meglátogatva: 2019.~november~4.]

%Internetes adatgyűjtés

\bibitem{web:GoogleEdge}
	\textbf{Edge TPU: Hands-On with Google’s Coral USB Accelerator},
	\newline{\footnotesize\url{heartbeat.fritz.ai/edge-tpu-google-coral-usb-accelerator-cf0d79c7ec56}}
	[meglátogatva: 2019.~október~23.]

\bibitem{web:Keras}
	\textbf{Keras Dokumentáció},
	\newline{\footnotesize\url{keras.io/}}
	[meglátogatva: 2019.~október~23.]

\bibitem{web:NNP}
	\textbf{Intel® Nervana™ Neural Network Processors},
	\newline{\footnotesize\url{www.intel.ai/nervana-nnp/}}
	[meglátogatva: 2019.~november~07.]

%\bibitem{web:Raoblog2019}
%	Naveen Rao,
%	\textbf{‘‘Accelerating with Purpose'' for AI everywhere},
%	2019,
%	\newline{\footnotesize\url{www.intel.ai/accelerating-for-ai/}}
%	[blog, meglátogatva: 2019.~November~07.]

%\bibitem{web:Raoblog2017}
%	Naveen Rao,
%	\textbf{Intel® Nervana™ Neural Network Processors (NNP) Redefine AI Silicon},
%	2017,
%	\newline{\footnotesize\url{www.intel.ai/intel-nervana-neural-network-processors-nnp-redefine-ai-silicon/}}
%	[blog, meglátogatva: 2019.~November~07.]

\bibitem{web:IntelNews2019}
	\textbf{At Hot Chips, Intel Pushes ‘AI Everywhere’},
	2019,
	\newline{\footnotesize\url{newsroom.intel.com/news/hot-chips-2019/}}
	[meglátogatva: 2019.~November~07.]

\bibitem{slide:nnpt}
	Andrew Yang,
	\textbf{Deep Learning Training At Scale	Spring Crest Deep Learning Accelerator (Intel® NervanaTM NNP-T)},
	2019,
	\newline{\footnotesize\url{www.slideshare.net/insideHPC/deep-learning-training-at-scale-spring-crest-deep-learning-accelerator/}}
	[letöltve: 2019.~november~07.]

\bibitem{yt:nnpt}
	Intel AI,
	\textbf{Introducing Intel Nervana Neural Network Processors for Training},
	2019,
	\newline{\footnotesize\url{www.youtube.com/watch?v=Wj4ifr3DDFg}}
	[megtekintve: 2019.~november~10.]

\bibitem{slide:nnpi}
	Ofri Wechsler \& Michael Behar \& Bharat Daga,
	\textbf{Spring Hill (NNP-I 1000) Intel’s Data Center Inference Chip},
	2019,
	\newline{\footnotesize\url{www.slideshare.net/insideHPC/spring-hill-nnpi-1000-intels-data-center-inference-chip/}}
	[letöltve: 2019.~november~07.]

\bibitem{yt:nnpi}
	Intel AI,
	\textbf{Introducing Intel Nervana Neural Network Processors for Inference},
	2019,
	\newline{\footnotesize\url{www.youtube.com/watch?v=fTMIjQePWog}}
	[megtekintve: 2019.~november~10.]

	%Github

\bibitem{github:nGraph}
	\textbf{NervanaSystems/ngraph: nGraph - open source C++ library, compiler and runtime for Deep Learning},
	GitHub repository,
	2019,
	\newline{\footnotesize\url{github.com/NervanaSystems/ngraph}}

\bibitem{github:PlaidML}
	\textbf{plaidml/plaidml: PlaidML is a framework for making deep learning work everywhere},
	GitHub repository,
	\newline{\footnotesize\url{github.com/plaidml/plaidml}}
	[meglátogatva: 2019.~október~20.]

	%Wikipedia

\bibitem{ wiki:constfold}
	\textbf{Constant folding --- {Wikipedia}{,} The Free Encyclopedia},
	2019,
	\newline{\footnotesize\url{en.wikipedia.org/w/index.php?title=Constant_folding&oldid=914455114}}
	[meglátogatva: 2019.~október~09.]
	
	\bibitem{wiki:plaidml}
	\textbf{PlaidML --- {Wikipedia}{,} The Free Encyclopedia},
	2019,
	\newline{\footnotesize\url{en.wikipedia.org/w/index.php?title=PlaidML&oldid=898300482}}
	[meglátogatva: 2019. október 5.]
	
	\bibitem{wiki:relu}
	\textbf{Rectifier (neural networks) --- {Wikipedia}{,} The Free Encyclopedia},
	2019,
	\newline{\footnotesize\url{ en.wikipedia.org/w/index.php?title=Rectifier_(neural_networks)&oldid=923288576}}
	[Meglátogatva: 2019.~október~29.]
	
	\bibitem{wiki:TPU}
	\textbf{Tensor processing unit --- {Wikipedia}{,} The Free Encyclopedia},
	2019,
	\newline{\footnotesize\url{en.wikipedia.org/w/index.php?title=Tensor_processing_unit&oldid=923241091}}
	[Meglátogatva: 2019. november 4.]

\end{thebibliography}