%-------------------------------------------------------------------------------
\chapter*{Bevezetés}\addcontentsline{toc}{chapter}{Bevezetés}
%-------------------------------------------------------------------------------
%DONE
A Deep learning avagy a mélytanulás a gépi tanulás neurális hálózatokat alkalmazó technikája napjaink egyik legnépszerűbb technológiája, melynek fejlesztését számos kutató intézmény és nagyvállalat végzi.
Megjelent a polgári életben is. Felhőalapú alkalmazások hátterében működik, többek között a Google online szolgáltatásaiban és már alkalmazzák a hordozható eszközök, táblagépek és mobiltelefonok biometrikus személyazonosításra. 

A mélytanulás használata rengeteg számítási kapacitást igényel ---ez bizonyos alkalmazások esetén költséges és nagy méretű számítógépeket jelent--- így sok helyen kiszorul a használata, illetve telemetria formájában érhető el csak.
Az 5G-nek hála, komolyabb alkalmazásokhoz is felhasználható lesz a felhő technológián működő gépi tanulás.
Ennek ellenére igény volna arra, hogy helyben elérhető legyen ez a technika. 
Ilyen lehet az orvosi alkalmazás, ahol számít a magas rendelkezésre állás vagy az autonóm robotok és önvezető autók, melyeknek bizonyos helyzetekben ott is kell működniük, ahol nincs rádiókapcsolat vagy internet elérés, nem is beszélve a hordozható eszközök olyan funkcióiról melyek használata frekventált.

Mikor fellendült a kutatása, legjobb hardverek erre a feladatra a fejlett grafikus kártyák voltak, melyek processzorainak számítási kapacitása és utasításkészlete alkalmassá tette, hogy a neurális hálózatokkal kapcsolatos számításokat hatékonyan végezze.
Azonban az iparban megjelentek speciális hardverek kifejezetten neurális hálózatok futtatására optimalizálva, hogy ki tudják elégíteni a megnövekedett számítási igényt, amit a technológia egyre szélesebb körű bevezetése generál.
A fenntarthatóság végett azonban nem szabad kihasználatlanul hagyni a már meglévő erőforrásokat. Témavezetőm, Dr. Kovács László projektje, a HuSSar nevet viselő hibrid architektúrájú szuperszámítógép is részben ebből az indíttatásból született. A HuSSar olyan hardverekből tevődik össze, melyek szerverek komponenseként régóta ott van az iparban. Egyedi hibrid architektúrája lehetőséget ad arra, hogy a neurális hálózatokkal kapcsolatos különféle számításokat olyan processzoron futtassuk, melyek azt optimálisan képesek végrehajtani így jelentős teljesítménynövekedés érhető el vele.
Ehhez szükséges még egy olyan keretrendszer, mely képes ezeket a számításokat ekképpen optimalizálni.

A Deep learning a szemem láttára fejlődött ki a kezdeti kísérletekből, a mindennapi életben is használt csúcstechnológiává. Úgy érzem leendő szakemberként most van arra alkalmam, hogy közelebbről is megismerkedjek vele, kivehessem részem a fejlesztésében. Észrevettem, hogy az ipar is nagy erőkkel fejleszti, ezért úgy vélem, hogy ez a tudás számomra nagyon jövedelmező lehet a munkaerőpiacon is.
Témavezetőm fejlesztésével, a HuSSar-ral az egyik általa tartott egyetemi kurzus során találkoztam, mikor azt megmutatta nekünk. Beszélt az eszköz felépítéséről és arról, milyen célból kezdte a fejlesztést. Továbbá látom unokaöcsém sikereit, aki ezen a területen kutat. Ezek miatt éreztem úgy, hogy ebben a témában szeretnék dolgozni, ha lehet az egyetemi tanulmányaim után is.

Ebben a szakdolgozatban szeretnék beszámolni, mit sikerült megtudnom a Deep Learning-ről, milyen új megvalósítások születtek az iparban és hogyan boldogultam ezekkel a technológiákkal. Eredeti célkitűzésem az Intel fejlesztés alatt álló \emph{nGraph} nevű környezetének fordítása és telepítése volt a fentebb említett HuSSar-ra. Ez a keretrendszer kifejezetten a neurális hálózatok olyan módú futtatására lett fejlesztve, ahol a hardver több típusú processzort tartalmaz.
Ezzel szerettük volna, ha sikerül a mélytanulás során alkalmazott neurális hálózatokat az összes processzortípuson elosztottan tanítani és futtatni. Hosszas próbálkozás után sem sikerült ez ügyben eredményt elérni, azonban a munka során megismerkedtem más az Intel által fejlesztett és fejlesztés alatt álló eszközeivel.
%-------------------------------------------------------------------------------
% Kész a Bevezetés
%-------------------------------------------------------------------------------