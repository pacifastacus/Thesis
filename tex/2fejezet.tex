%-------------------------------------------------------------------------------
\chapter{Mély tanuló alkalmazások fejlesztése}
%-------------------------------------------------------------------------------
% TODO: Mélytanulásos keretrendszerek bemutatása általánosságban (mire jók, stb...)
A mély tanulás, azon belül is a mesterséges neurális hálózatok gyors fejlődősét és részben népszerűségét a hozzá készített programozási keretrendszereknek köszönheti, melyek java szabad hozzáférésű. Ezek a keretrendszerek arra hivatottak, hogy támogassák a neurális hálózatok fejlesztését új programozási módszertant adva. Már létező programozási nyelvekre épülnek, leginkább python-ra. Ezekben a keretrendszerekben egyszerűen implementálhatunk neurális hálózatokat úgy, hogy egyfajta nyelvi eszközkészletet adnak a hálózat definiálására.

\section{TensorFlow}

%TODO Nincs Befejezve
\section{Keras}

A Keras egy python nyelven írt programkönyvtár, vagy ahogy önmagát hívja ''magas szintű neurális hálózat API''\cite{web:Keras}. Érdekessége, hogy más olyan keretrendszerekkel együttesen használható, amelyek a Keras-hoz hasonlóan magas absztrakciós szinten biztosítják a hálózatok implementálását, mint például a TensorFlow. A Keras-ról szerzett tudásom javát Chollet könyvéből és a keretrendszer dokumentációjából szereztem.\cite{Chollet}\cite{web:Keras}.

\subsection{Neurális hálózat definiálása}

Keras-ban egy neurális hálózatot \emph{model}-nek hívunk (gyakran máshol is így neveznek egy konkrét neurális hálózatot). Egy \emph{model} létrehozásához a \verb|Keras.models| modulban definiált metódusokkal lehetséges. A keretrendszerben az adatokat tenzorokként kell reprezentálnunk, ezért érdemes a \emph{numpy}\footnote{lásd:\url{https://numpy.org/}} nevű python csomaggal együtt használni. A következő kódokban szeretném szemléltetni a Keras használatának módját.

A \ref{lst:defLayers} kódlista egy két rétegből álló neurális hálózat definiálását mutatja be.
\begin{minipage}{\linewidth}
\begin{lstlisting}[language=Python, caption=Két Neurális hálózat rétegeinek definiálása]
from keras import models
from keras import layers

network = models.Sequential()
network.add(layers.Dense(32, activation='relu', input_shape=(784,)))
network.add(layers.Dense(10, activation='softmax'))
\end{lstlisting}\label{lst:defLayers}
\end{minipage}
Ezen lista 4. sorában inicializáljuk a hálózatot, az utána következő sorokban új neuron rétegeket adunk a hálózathoz. A Keras-ban különböző előre definiált rétegkapcsolatok vannak. Itt a \verb|layers.Dense| osztály egy sűrűn kapcsolt réteget implementál. %TODO sürün kapcsolt réteg definíciója
A réteg bemenetként $28 *28 = 784$ elemű vektorokból álló mátrixot tud fogadni, másik dimenziója nem meghatározott. Ez a kötegelt adatfeldolgozás szempontjából fontos, tehát a bemeneti vektorok sorozatát egy pontosan egy dimenziójában tetszőleges mátrixszal definiáljuk. A réteg kimenete egy 32 dimenziós vektor. A következő rétegnél nem adtuk meg a bemenet méretét, a keretrendszerben ez implicit módon rendelődik a réteghez. 

Mindkét réteghez tartozik egy \emph{activation} paraméter, mely string típus kell legyen, és a réteg neuronjaihoz tartozó aktivációs függvényt adja meg. A példában a \verb|'relu'| string a $ReLu(x) = max\{0,x\}$ függvényt jelenti, a \verb|'softmax'|, pedig egy eloszlásfüggvény, a kimeneti vektoron. %TODO softmax ponhtosítása

A \verb|Keras.layers| modulban A \verb|Dense| osztályon kívül implementálva vannak konvolúciós, összefésülő, zaj, stb. rétegek, melyek a fenti módon tetszés szerint egymásra szervezhetőek a keretrendszerben, így szinte tetszőleges hálózat alakítható ki.

\subsection{Hálózat betanítása és következtetés futtatása}

A megalkotott hálózatot a \verb|network| objektum definiálja. Hogy tanítható legyen el kell látni egy optimizálóval és egy veszteségfüggvénnyel, melyek együtt adják a hálózatot betanító algoritmust. A \verb|compile()| metódus ,,összerakja'' a hálózatot a betanítóval, a \verb|fit()| pedig elvégzi a tanítást, és au \emph{epoch}-ok számának megfelelő méretű tömböket tartalmazó objektum referenciájával tér vissza, mely tömbökben össze vannak gyűjtve a veszteségfüggvény értékei és a felhasználói metrikák. 

\begin{minipage}{\linewidth}
\begin{lstlisting}[language=Python,caption=Hálózat betanítása]
network.compile(optimizer = 'rmsprop',
loss = 'binary_crossentropy',
metrics = ['acc'])

history = network.fit(train_datas,
					train_labels,
					epochs=20,
					batch_size=512)
\end{lstlisting}\label{lst:fitNetwork}
\end{minipage}

A \verb|train_datas| és \verb|train_labels| a betanító adatokat és azok címkéit tartalmazó tömbök. Ezek elemszáma kötelezően meg kell egyezzen. A \emph{batch\_size} paraméter az egy \emph{epoch}-ban egyszerre feldolgozandó adatokat jelenti. A betanítás végén a \verb|network| egy \emph{betanított}, a célfeladat megoldására felhasználható neurális hálózat lesz. Alkalmazásához a \verb|predict()|metódus használatos,mely a paraméterként megadott adatokhoz tartozó valószínűség értékekkel tér vissza.
%TODO 
\begin{minipage}{\linewidth}
\begin{lstlisting}[language=Python, caption=Következtetés]
	result = network.predict(datas)
\end{lstlisting}
\end{minipage}

\subsection{Hatékonyságvizsgálat}


%TODO Nincs Befejezve