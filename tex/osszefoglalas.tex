%-------------------------------------------------------------------------------
\chapter*{Összefoglalás}\addcontentsline{toc}{chapter}{Összefoglalás}
%-------------------------------------------------------------------------------
%\lipsum[1]
A gépi tanulás olyan területeken jelent meg melyek új kihívásokat szültek. A hardvergyártók kifejezetten deep learningre optimalizált megoldásokat kínálnak és a jövőbeli fejlesztések is ebbe az irányba mutatnak. Ezen dolgozatomban bemutatott, mély tanulásban használatos technológiákon látható, hogy egyre változatosabb deep learning rendszereket alapját tudják nyújtani. Olyan keretrendszerek születtek független fejlesztőcsapatok és a hardvergyártók jóvoltából, melyekkel gyorsan és hatékonyan végezhető mély tanuló alkalmazások fejlesztése. Ennek jóvoltából a mindennapi életben is megjelenik ez a technológia, és azon is túl a kutatásokban megjelenő nehezen megoldható számítási problémák váltak kivitelezhetővé. Ezen technológiákat felölelő szakterületek a munkaerőpiacon is igen keresetté válhatnak.

A hardveripar most is sok erőforrást fektet a gyakorlatban használt mesterséges neurális hálózatok hatalmas számítási igényének kielégítésére. Újabb és hatékonyabb eszközöket kínálnak erre a feladatra. Némelyikük a korábban bevált GPGPU architektúrák továbbfejlesztése, míg mások speciális, csakis kifejezetten deep learning architektúrák. Ezek az eszközök a több magos masszívan párhuzamos számításokra alapoznak, melyek jól alkalmazhatóak a neurális hálózatokon.

Ezzel párhuzamosan hibrid számítási rendszereket támogató keretrendszerek is készülnek, szintén a párhuzamos számítás modellre, azontúl az egyes architektúrák eltérő működésére alapozva. Ennek köszönhetően a deep learning alkalmazások változatos hardver architektúrákon futtathatóak, melyek a nyers erőn túl a számítások hatékony végrehajtásával érnek el teljesítménynövekedést. Az ilyen technológiák továbbá lehetőséget biztosítanak, hogy a felhasználók meglévő hardvereiket építsék be deep learning rendszerekbe. Ez a fejlesztési költségek szempontjából nagyon előnyös, ezért a jövőben várhatóan egyre több szerveren találkozhatunk tanuló vagy tanítható alkalmazásokkal.

A nagygépes világgal párhuzamosan megjelentek külön mély tanulásra specializált integrált áramkörök a hordozható és személyi eszközökön is. A Deep Learning így egy ,,kézzel fogható'' technológiává válik. Ez az irány elősegíti a valós idejű mély tanuló alkalmazások fejlesztését, mint például az önvezető járművek, intelligens gyártósorok, személyi applikációk vagy egyes orvosi alkalmazások. Úgy látszik a jövőben erre a szolgáltatók is számítanak: ezen speciális hardverek némelyike építőkövei lehetnek az ún. \emph{Edge Computing} modellnek: míg kezdetben a felhőszolgáltatásokat nyújtó vállalatok neurális hálózatai teljes egészében szervereiken futottak, mára kezd kibontakozni az az elképzelés, hogy ezeket a számításokat a felhasználóhoz egészen közel lehet vinni, így közel valós idejűek lehetnek a felhő alapú deep learning szolgáltatások is a jövőben.

A szakdolgozatom írása során rengeteg új ismeretre tettem szert. Igyekeztem a legmodernebb hardvereket és legújabb szoftverfejlesztéseket felkutatni. Így szembesültem azzal is, hogy ennek a szakterületnek csak felszínét vizsgáltam még. Ahhoz, hogy tényleg jó rálátásom legyen a jövőbeli fejlesztésekre, a gépi tanulással kapcsolatos ismereteimet mesterképzésben szeretném elmélyíteni.

Az nGraph lefordításának sikertelensége témavezetőm hibrid rendszerére nem tántorított el a korábban megfogalmazott terv kivitelezésétől. A keretrendszer élénk fejlesztését látva van esély rá, hogy a későbbiekben újból megkíséreljük lefordítani a HuSSarra, hogy öszevethessük az nGraph teljesítményét más hibrid rendszereken működő fordítókéval.