\documentclass[magyar,12pt,oneside]{article}
\usepackage[utf8]{inputenc}
\usepackage[T1]{fontenc}
\usepackage[magyar]{babel}
\usepackage{enumitem}	%felsorolások extra paraméterezése
\usepackage[top=3cm,bottom=3cm,left=3cm,right=2cm]{geometry}
\usepackage{graphicx}
%\usepackage{mathptmx}	% Times betűtípus
\usepackage{fontspec}		% csomag a \setmainfont hoz (XeLatex!!!!)
\usepackage[]{hyperref}

\setmainfont{Times New Roman}	%Times New Roman betűtípus

\title{A diplomadolgozat formai követelményei \\ Debreceni Egyetem Informatikai kar}
\author{}
\date{}

\hypersetup{
	pdftitle={Your title here},
	pdfauthor={Your name here},
	pdfsubject={Your subject here},
	pdfkeywords={keyword1, keyword2},
	bookmarksnumbered=true,     
	bookmarksopen=true,         
	bookmarksopenlevel=1,       
	colorlinks=true,            
	pdfstartview=Fit,           
	pdfpagemode=UseOutlines,    % this is the option you were lookin for
	pdfpagelayout=TwoPageRight
}

\renewcommand{\familydefault}{\sfdefault}	%betűcsalád beállítása az egész dokumentumra


\begin{document}
\maketitle
%\sffamily
\section[Szerkezeti felépítés]{A diplomadolgozat szerkezeti felépítése}

A diplomadolgozat szerkezete az alábbiak szerint tagolódjék
\begin{itemize}
\item Cím
\item Tartalomjegyzék
\item Bevezetés
\item Tárgyalási (leíró, elemző) rész
\item Összefoglalás
\item Irodalomjegyzék
\item Függelék
\item Köszönetnyilvánítás (amelynek helye a dolgozatban tetszőlegesen elhelyezhető)
\end{itemize}


\section[Formai követelmények]{A diplomadolgozat készítésének formai követelményei}

\subsection[Borítólap]{Borítólap kivitele és felirata}

Belső címlap feliratai
\begin{itemize}
\item Egyetem, kar, tanszék (lap tetején, középre rendezve)
\item A diplomamunka/ szakdolgozat címe (felső harmadban, középen)
\item A szerző neve, szak megnevezése (alsó harmadban jobb oldalon)
\item Konzulens neve, beosztása (alsó harmadban bal oldalon)
\item A dolgozat benyújtásának helye (Debrecen) és éve(lap alján, középre rendezve)
\end{itemize}
A tartalomjegyzéket a belső címlap utáni oldalon decimális számrendszerben kell közölni.

\subsection[Formátum]{Formátum}

\begin{description}[style=nextline]
\item[Margó]
A szövegszerkesztés megkezdése előtt szükséges a margók méretének beállítása. Méretezés: bal margó 3 cm, jobb 2 cm, felső 3 cm, alsó 3 cm.
\item[Betűtípus]
Választáskor a legfőbb szempont az olvashatóság legyen. A betűkészlet tartalmazza a magyar mellékjeles betűket különös tekintettel az ő és ű betűkre. Betűtípus: Times New Roman.
\item[Betűméret]
A dolgozat szövegét legcélszerűbb 12 pontos betűvel nyomtatni. A nagyobb, 14-16 pontos betűk a címlap készítésekor és a fejezetcímek írásakor használhatók.
\item[Sortávolság]
Kiválasztásakor a fő szempont az olvashatóság legyen. Az optimális sorköz a másfeles (1,5-es) méret.
\end{description}

\subsection[Tagolás]{A diplomadolgozat ajánlott tagolása}
\begin{description}[style=nextline]
\item[A cím]
Legyen rövid, érthető és a tartalmat jól kifejező.
\item[Tartalomjegyzék]
A tartalomjegyzék a részek, a fejezetek, az alfejezetek és a még kisebb egységek címéből, valamint a hozzájuk tartozó oldalszámokból épül fel. A tartalomjegyzékben a főszövegen kívül a járulékos részeket is fel kell tüntetni.
\item[Bevezetés]
Indokolja a témaválasztást, annak elméleti és gyakorlati jelentőségét, fogalmazza meg a célkitűzéseket és a vizsgálati módszereket (2-3 oldal).
\item[Tárgyalási rész (30-40 oldal)]
Az érdemi rész tartalmát a témavezető és a hallgató(k) határozzák meg.\\
A téma kifejtése. A hazai szakirodalom feldolgozása mellett emeli a diplomadolgozat értékét, ha a szerző külföldi forrásokra is támaszkodik. Felhívjuk a figyelmet az Internet által adott lehetőségekre, valamint a kari könyvtárban fellelhető külföldi folyóiratokra.
\item[Összefoglalás (2-3 oldal)]
Röviden, tömören és világosan ismertetni kell a fontosabb megállapításokat és következtetéseket. Bátran és egyértelműen közölni kell, ha valamely célkitűzést nem sikerült megvalósítani. (Megfelelő indoklás mellett ez egyáltalán nem csökkenti a diplomadolgozat értékét!)
\item[Irodalomjegyzék]
Az irodalomjegyzék tartalmaz minden olyan könyv-, periodika-, elektronikus média-hivatkozást, amelyeknek a felhasználásával készült a munka. Ez nem jelenti azt, hogy minden egyes munkából történt idézet, de azt igen, hogy ezeket az irodalmakat felhasználta a szerző.\linebreak
A javasolt tagolás az alábbi (megfelelő számú irodalom esetében célszerű magyar és külföldi részekre bontani)
\begin{itemize}
\item könyvek, cikkek, tanulmányok
\item statisztikai adatforrások
\item jogszabályok
\item INTERNETES adatgyűjtés
\end{itemize}
(ebben az esetben a hivatkozásokra különös figyelmet kell fordítani, a hivatkozásokat dokumentum mélységig kell megadni).
\item[Függelék (maximum 8-10 oldal)]
Ide kerülnek azok a nagyobb méretű táblázatok, ábrák.\\
Ide helyezhető el továbbá a kérdőíves felmérés alapjául szolgáló dokumentumok, továbbá a statisztikai és matematikai számítások alaptáblái is.\\
Egyes esetekben rövidebb szöveges dokumentumok (pl. szerződése, jogszabály részletek, stb.) is helyet kaphatnak itt.
\item[Köszönetnyilvánítás]
A hallgató(k) köszönetet nyilvánít(anak) mindazoknak, akiktől (elméleti, gyakorlati, erkölcsi, stb.) segítséget kapott.

\end{description}

\textbf{A PLÁGIUM NYILATKOZATOT NEM KELL BEKÖTTETNI A KÉSZ SZAKDOLGOZATBA!}

\appendix

\begin{figure}[!ht]
	\includegraphics[width=\linewidth,keepaspectratio]{cimlap_kulso.png}
	\caption{Külső Címlap vagy borító}
\end{figure}
\begin{figure}[!ht]
	\includegraphics[width=\linewidth,keepaspectratio]{cimlap_belso.png}
	\caption{Belső címlap}
\end{figure}

\end{document}